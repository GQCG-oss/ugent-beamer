% Aspectratio 16:9 should be used
% The theme is not suited for 4:3 aspectratio
\documentclass[aspectratio=169]{beamer}

% Metadata of the presentation
\title{UGent beamer theme}
\subtitle{Some optional subtitle}
\subtitle{Powered by \LaTeX}
\date[ISBT 2018]{27th International Symposium of Beamer Themes}
\author[DB]{Dries Benoit --- \texttt{dries.benoit@ugent.be}}

% Load the UGent theme
%\usetheme[language=en,faculty=eb,usecolors]{ugent}
\usetheme[language=en,usecolors]{ugent}

% Have this if you'd like section slides 
\AtBeginSection[]{
    \sectionframe
}

\begin{document}

% Have this if you'd like the presentation to start 
% with a large UGent logo
\logoframe

% I guess you always want a titleframe
\titleframe

% Have this if you'd like a frame containing the 
% table of content
\begin{frame}{Overview}
    \tableofcontents[hideallsubsections]
\end{frame}

% Start of the first section
\section{Fonts and layout}

\begin{frame}
    \frametitle{This is the frame title}
    \framesubtitle{With optional frame subtitle}
    Regular text in the body of the slide is black and rendered in Helvetica sans-serif.\\[.5cm]
    Helvetica has no corresponding math font.
    Therefore, equations are typeset in Computer Modern sans-serif and are also displayed in plain black:
    \begin{equation*}
        F(x|\mu,s) = \int_{-\infty}^x s^{-1}\left(1+e^{-\frac{v-\mu}{s}}\right)^{-2} e^{-\frac{v-\mu}{s}}\;\mathsf{d}v = \frac{1}{1+e^{-\frac{x-\mu}{s}}}
    \end{equation*}
    Emphasis can be added by using \textbf{bold} typeface, \textit{italic}, {\color{ugent-alert}colors} or {\color{ugent-alert}\textbf{\textit{any combination}}}.\\
    More about colors follows later.
\end{frame}

\begin{frame}
    \frametitle{The frame title is rendered in Small Caps}
    The official UGent Powerpoint/Keynote templates have all titles in both ALL CAPS, \textbf{bold} and \underline{underline}.\\[.5cm]
    In my opinion, this combination is somewhat \underline{\textbf{AGGRESSIVE AND UNPLEASANT TO THE EYE}}.\\[.5cm]
    Instead, this theme makes use of \textsc{Small Caps} for all titles and subtitles
\end{frame}


\section{Lists and enumeration}

\begin{frame}
    \frametitle{Lists of items}
    This is how a list of unnumbered items looks:
    \begin{itemize}
        \item Item 1
        \item Item 2
        \item Item 3
    \end{itemize}
    \vspace{.25cm}
    Nested lists of items are possible too:
    \begin{itemize}
        \item Item 1
            \begin{itemize}
                \item Subitem a
                \item Subitem b
            \end{itemize}
        \item Item 2
            \begin{itemize}
                \item Subitem a
                \item Subitem b
            \end{itemize}
    \end{itemize}
\end{frame}

\begin{frame}
    \frametitle{Lists of items}
    This is how a list of numbered items looks:\\[.25cm]
    \begin{enumerate}
        \itemsep.5cm
        \item Item 1
        \item Item 2
        \item Item 3
        \item Item 4
        \item Item 5
    \end{enumerate}
\end{frame}


\section{Colors}

\begin{frame}
    \frametitle{Colors}
    \begin{itemize}
        \item The offical UGent colors (in RGB)  are part of the theme.
        \item The primary UGent color is {\color{ugentblue} ugentblue}, and the secondary color is {\color{ugentyellow} ugentyellow}.
        \item The faculty specific UGent themes can use the faculty color as secondary color.
            \begin{center}
                \begin{minipage}[t]{.35\textwidth}
                    \begin{itemize}
                        \item {\color{ugentblue}  \textbf{ugentblue}}
                        \item {\color{ugentyellow}\textbf{ugentyellow}}
                        \item {\color{ugent-lw}   \textbf{gent-lw}}
                        \item {\color{ugent-re}   \textbf{gent-re}}
                        \item {\color{ugent-we}   \textbf{gent-we}}
                        \item {\color{ugent-ge}   \textbf{gent-ge}}
                        \item {\color{ugent-ea}   \textbf{gent-ea}}
                    \end{itemize}
                \end{minipage}%
                \begin{minipage}[t]{.35\textwidth}
                    \begin{itemize}
                        \item {\color{ugent-eb} \textbf{ugent-eb}}
                        \item {\color{ugent-di} \textbf{ugent-di}}
                        \item {\color{ugent-pp} \textbf{ugent-pp}}
                        \item {\color{ugent-bw} \textbf{ugent-bw}}
                        \item {\color{ugent-fw} \textbf{ugent-fw}}
                        \item {\color{ugent-ps} \textbf{ugent-ps}}
                    \end{itemize}
                \end{minipage}
                \vspace{.5cm}
            \end{center}
        \item Note that every faculty color name refers to the abbreviation of the faculty name.
    \end{itemize}
\end{frame}

\begin{frame}
    \frametitle{More about colors}
    \begin{itemize}
        \itemsep.25cm
        \item The main colors of the presentation are ugentblue, black and white.
        \item The secondary color should only be used exceptionally.
        \item The theme defines a special color \texttt{ugent-alert}.
        \item This color is \emph{or} {\color{ugentyellow}ugentyellow} \emph{or} the faculty color.
        \item The special color \texttt{ugent-alert} equals the faculty color:
            \begin{itemize}
                \item \textbf{if} the faculty specific template is used
                \item \textbf{and if} the option \texttt{usecolors} is set 
            \end{itemize}
        \item In all other cases \texttt{ugent-alert} is {\color{ugentyellow}ugentyellow}
    \end{itemize}
\end{frame}

\begin{frame}
    \frametitle{Example of ugent-alert}
    \begin{theorem}
        There is no largest prime number.
    \end{theorem} 
    \pause
    \begin{block}{Proof}
        \begin{enumerate} 
            \item<2-| alert@2> Suppose $p$ were the largest prime number. 
            \item<3-| alert@3> Let $q$ be the product of the first $p$ numbers. 
            \item<4-| alert@4> Then $q+1$ is not divisible by any of them. 
            \item<5-6| alert@5> But $q + 1$ is greater than $1$, thus divisible by some prime number not in the first $p$ numbers.\only<6>\qed
        \end{enumerate}
    \end{block}
\end{frame}

\begin{frame}
    \frametitle{Another example of ugent-alert}
    Blocks can be created for definition, proofs, examples, etc.
    \begin{block}{Regular block}
        This is an important message.
    \end{block}
    \vspace{.5cm}
    \pause
    A special kind of block is the \texttt{alertblock}:
    \begin{alertblock}{Alert!}
        This is a very important message.
    \end{alertblock}
\end{frame}


\section{Frame numbers}

\begin{frame}
    \frametitle{Frame numbers}
    \begin{itemize}
        \itemsep.25cm
        \item By default frame numbers are places on every ``regular'' frame.
        \item That excludes logoframes, titleframes and sectionframes.
        \item The frame number is always followed by the total number of frames.
        \item The theme option \texttt{noframenumber} removes frame numbers on all slides.
    \end{itemize}
\end{frame}

\section{Questions?}

% End presentation with titleframe
\titleframe

\end{document}
